%	DOCUMENT TYPE
%\documentclass{revtex4-1}
%\documentclass[showkeys]{revtex4-1}
%\documentclass[aps,prb,preprint,linenumbers]{revtex4-1}
%\documentclass[aps,prb,reprint]{revtex4-1}
%\documentclass[aps,prl,preprint,linenumbers]{revtex4-1}
%\documentclass[aps,prl,reprint]{revtex4-1}

\documentclass[aps,preprint,notitlepage,nofootinbib]{revtex4-1}
%\documentclass[aps, preprint, prb, linenumbers, showpacs]{revtex4-1}

%	PACKAGES
\usepackage[utf8]{inputenc}
\usepackage{amssymb,amsmath,esint,amsfonts,mathtools,dsfont,bm}
\usepackage[usenames,dvipsnames]{xcolor}
\usepackage[pdftex]{graphicx}
\usepackage[pdftex,plainpages=false,colorlinks=true,linkcolor=Red, citecolor=blue, urlcolor=blue]{hyperref}
\usepackage[most]{tcolorbox}
\usepackage{empheq}


%	FONT
% \usepackage[fullfamily,lf,minionint,openg,loosequotes]{MinionPro}
%\usepackage[T1]{fontenc}
%\usepackage{microtype}
%\usepackage{times}
%\usepackage{mathptmx}
%\usepackage{newtxtext,newtxmath}
% \usepackage{txfonts}

\renewcommand{\vec}[1]{\bm{\mathrm{#1}}}

%% tensor

\usepackage{stackengine}\stackMath
\def\lvec{{\rotatebox{180}{$\mkern+0mu\mathchar"017E$}}}
\def\tensign{\smash{\stackon[-1.99pt]{\mkern-4mu\mathchar"017E}{\rotatebox{180}{$\mkern+0mu\mathchar"017E$}}}}
\def\tensor#1{\def\useanchorwidth{T}\stackon[-4.3pt]{#1}{\,\tensign}}







%	DOCUMENT
\begin{document}

\title{Quick Guide on Electronic DC Transport on Cuprates}
\author{S. \surname{Verret}}
\affiliation{D\'epartement de physique, Universit\'e de Sherbrooke, Qu\'ebec, Canada  J1K 2R1}
\date{\today}
\keywords{}
\begin{abstract}
I will write the abstract once the document has fully take form.
\end{abstract}

\maketitle


\section{Conductivity tensors}
\noindent
Three conductivity tensors $\tensor{\sigma}$, $\tensor{\alpha}$, and $\tensor{\beta}$ suffice to determine all transport coefficients. They are defined by the relations between the $x$,$y$,$z$ components of the electric current, $\vec j_e$, the heat current $\vec j_Q$, the electric field $\vec E$, and the Temperature gradient $\vec \nabla T$~\cite{ashcroft_solid_1976,arsenault_transport_2013}:
\begin{align}
\vec j_e &= \tensor{\sigma}\ \vec E + \tensor{\alpha}\ (-\vec \nabla T)\\
\vec j_Q &= \tensor{\alpha}'\ \vec E + \tensor{\beta}\ (-\vec \nabla T).
\end{align}
Note: these conductivity tensors are also often noted $\tensor{\sigma}=L_{11}$, $\tensor{\alpha}=L_{12}$, $\tensor{\beta}=L_{22}$. They depend on temperature $T$ and magnetic field $\vec B$, satisfying the Onsager relations:
\begin{align}
\tensor{\sigma}(\vec B) &= \tensor{\sigma}^{\intercal}(-\vec B)\\
\tensor{\alpha}(\vec B) &= -T\tensor{\alpha}'^{\intercal}(-\vec B)\\
\tensor{\beta}(\vec B) &= \tensor{\beta}^{\intercal}(-\vec B).
\end{align}
The second Onsager relation explains why we do not need to worry about $\tensor{\alpha'}=L_{21}$.

In almost all cases, these tensors can be computed as energy integrals of the same kernel $\sigma_{ij}(\epsilon)$, which can be viewed as a \emph{density of conductivity}. The specific heat $C_V$ can also be computed with such an integral, but of the density of states $N(\epsilon)$ in the place of $\sigma_{ij}(\epsilon)$. With $i,j$ denoting $x$,$y$,$z$ components of the tensors, these integrals are:
\begin{align}
\sigma_{ij}
&=
\int_{-\infty}^{\infty} d\epsilon
\Big(-\frac{\partial f(\epsilon)}{\partial \epsilon}\Big)\sigma_{ij}(\epsilon)
\label{sigma}
\\
\alpha_{ij}
&=
\int_{-\infty}^{\infty}d\epsilon
\bigg[\Big(-\frac{\partial f(\epsilon)}{\partial \epsilon}\Big)\frac{\epsilon}{T}\bigg]\frac{\sigma_{ij}(\epsilon)}{-e}
\label{alpha}
\\
\beta_{ij}
&=
\int_{-\infty}^{\infty}d\epsilon
\bigg[\Big(-\frac{\partial f(\epsilon)}{\partial \epsilon}\Big)\frac{\epsilon^2}{T}\bigg]\frac{\sigma_{ij}(\epsilon)}{e^2}
\label{beta}
\\
C_V
&=
\int_{-\infty}^{\infty}d\epsilon
\bigg[\Big(-\frac{\partial f(\epsilon)}{\partial \epsilon}\Big)\frac{\epsilon^2}{T}\bigg]N(\epsilon).
\end{align}
where the electron charge is $-e$. Thus, $\sigma_{ij}(\epsilon)$ act as a density of conductivity, which is ``sampled'', in these integrals, by derivatives of the Fermi-Dirac distribution~$f(\epsilon)$:
\begin{align}
f(\epsilon) = \frac{1}{e^{(\epsilon-\mu)/k_{B}T}+1}.
\end{align}
where~$\mu$ is the chemical potential.

The remaining sections of this document explain how to compute $\sigma_{ij}(\epsilon)$ in various approximations.
Before to do so, however, let us define the usual measurable transport quantities, and give some examples of how ``to think'' about the above integrals.

\subsection{Resistivity \& Hall effect}

The resistivity tensor $\tensor{\rho}$ is defined as $\tensor{\sigma}^{-1}$.
\begin{align}
\tensor{\rho} &\equiv \tensor{\sigma}^{-1}
\end{align}
For example, in the 2D case, $\tensor{\rho}$ is a $2\times2$ inverse:
\begin{align}
\tensor{\rho}
&=
\begin{pmatrix}
\sigma_{xx} & \sigma_{xy}\\
\sigma_{yx} & \sigma_{yy}
\end{pmatrix}^{-1}
=
\frac{1}{\sigma_{xx}\sigma_{yy}-\sigma_{xy}\sigma_{yx}}
\begin{pmatrix}
\sigma_{yy} & -\sigma_{xy}\\
-\sigma_{yx} & \sigma_{xx}
\end{pmatrix}
\end{align}
The longitudinal resistivity is then given as:
\begin{align}
\rho_{xx} 
&= 
\frac{\sigma_{yy}}{\sigma_{xx}\sigma_{yy} - \sigma_{xy}\sigma_{yx}}
\approx
\frac{1}{\sigma_{xx}},
\end{align}
where $\sigma_{xy}$ and $\sigma_{yx}$ are usually only non-zero under magnetic field $\vec B$, are usually very small, and give rise to the transverse resistivity:
\begin{align}
\rho_{xy} 
&= 
\frac{-\sigma_{xy}}{\sigma_{xx}\sigma_{yy} - \sigma_{xy}\sigma_{yx}},
\approx
\frac{-\sigma_{xy}}{\sigma_{xx}\sigma_{yy}}.
\end{align}
From the latter, the Hall coefficient, $R_H$ and the Hall number, $n_H$, are defined as:
\begin{align}
R_H = \frac{-\rho_{xy}}{B}
&\qquad
n_H = \frac{1}{-eR_H}.
\end{align}



\subsection{Thermoelectricity: Seebeck, Pelletier \& Nernst effects}

The thermoelectricity tensor $\tensor{Q}$ is defined as:
\begin{align}
\tensor{Q} &\equiv \tensor{\sigma}^{-1}\tensor{\alpha}
\end{align}
For example, in the 2D case, the $2\times2$ matrix for $\tensor{Q}$ is:
\begin{align}
\tensor{Q}
&=
\begin{pmatrix}
\sigma_{xx} & \sigma_{xy}\\
\sigma_{yx} & \sigma_{yy}
\end{pmatrix}^{-1}
\begin{pmatrix}
\alpha_{xx} & \alpha_{xy}\\
\alpha_{yx} & \alpha_{yy}
\end{pmatrix}
=
\begin{pmatrix}
  \frac{\sigma_{yy}\alpha_{xx}-\sigma_{xy}\alpha_{yx}}
  {\sigma_{xx}\sigma_{yy}-\sigma_{xy}\sigma_{yx}} 
& \frac{\sigma_{yy}\alpha_{xy}-\sigma_{xy}\alpha_{yy}}
  {\sigma_{xx}\sigma_{yy}-\sigma_{xy}\sigma_{yx}}\\
  \frac{\sigma_{xx}\alpha_{yx}-\sigma_{yx}\alpha_{xx}}
  {\sigma_{xx}\sigma_{yy}-\sigma_{xy}\sigma_{yx}}
& \frac{\sigma_{xx}\alpha_{xy}-\sigma_{yx}\alpha_{yy}}
  {\sigma_{xx}\sigma_{yy}-\sigma_{xy}\sigma_{yx}}
\end{pmatrix}
\end{align}


\subsection{Thermal conductivity \& Righi-Leduc effect}
In the same way, the thermal conductivity $\tensor{\kappa}$ can be obtained from:
\begin{align}
\tensor{\kappa} &\equiv -\tensor{\beta} + T\tensor{\alpha}\tensor{\sigma}^{-1}\tensor{\alpha}
\end{align}

\section{Semi-classical case}
\noindent
To get full magnetic field dependence in $\sigma_{ij}(\epsilon)$, one must use the Chambers Formula~\cite{ashcroft_solid_1976}:
\begin{align}
\sigma_{ij}(\epsilon) 
&=
e^2\iiint_{\text{BZ}}\frac{d^3k}{(2\pi)^3}
v_{i}(\vec k)\bar{v}_{j}(\vec k)
\delta\big(\epsilon-E(\vec k)\big),
\label{chambers1}
\end{align}
which is an integral on all $\vec k$-states in the first Brillouin zone (BZ). The Dirac delta indicates that each states only contributes to this ``density of conductivity'' at the energy~$E(\vec k)$. The weight of this contribution is given by a component of the group velocity~$\vec v(\vec k)$:
\begin{align}
v_{i}(\vec k) = \frac{1}{\hbar}\frac{\partial E(\vec k)}{\partial k_i},
\label{velocity}
\end{align}
multiplied by a ``decaying'' or ``travelling'' velocity, $\bar{v}_{j}(\vec k)$:
\begin{align}
\bar{v}_{j}(\vec k)
&=
\int_{0}^{\infty}dt
\exp\Big[
-\int_{0}^{t}dt'/\tau_{\vec k(t')}
\Big]
v_{j}(\vec k(t)),
\label{chambers2}
\end{align}
where we find the lifetime $\tau_{\vec k}$ (which is not only $\vec k$-dependent, but can also be $\epsilon$- and $T$-dependent, as discussed in section~\ref{section_lifetime}). Both $\tau_{\vec k}$ and $v_j(\vec k)$ are evaluated on a trajectory $\vec k(t)$, in $\vec k$-space. This trajectory starts at $\vec k(0)=\vec k$, and is governed by the semiclassical differential equation of motion:
\begin{align}
\hbar\frac{d\vec k(t)}{dt} =
-e \big(\vec v(\vec k(t))\times\vec B \big)
\label{semiClassicalMovement}
\end{align}
where $\vec B$ is the magnetic field. To compute conductivity tensors, one must thus solve the equation of motion \eqref{semiClassicalMovement} for each starting point $\vec k$ contained the integral of \eqref{chambers1}. Some simplifications are helpful.

\begin{tcolorbox}[colback=black!2!white,colframe=white!10!black,title=All-in-one equation]
At this point, we can reduce all the tensors into a single expression. We substitute \eqref{chambers2} and \eqref{velocity} in \eqref{chambers1}, and then substitute the result in \eqref{sigma}, \eqref{alpha}, and \eqref{beta}, unifying them in the following monster:
\begin{align}
\sigma_{ij}^{(n)}
&=
\int_{-\infty}^{\infty}d\epsilon
\bigg[\Big(-\frac{\partial f(\epsilon)}{\partial \epsilon}\Big)\frac{\epsilon^{n}}{T^{n}}\bigg]
(-e)^{2-n}\iiint_{\text{BZ}}\frac{d^3k}{(2\pi)^3}
\frac{1}{\hbar}\frac{\partial E(\vec k)}{\partial k_i} %v_{i}(\vec k)
\nonumber\\
&\qquad\times
\int_{0}^{\infty}dt
\exp\Big[
-\int_{0}^{t}dt'/\tau_{\vec k(t')}
\Big]
\frac{1}{\hbar}\frac{\partial E(\vec k)}{\partial k_j}\Big|_{\vec k(t)} %v_{j}(\vec k(t))
\ \delta\big(\epsilon-E(\vec k)\big)
\label{chambers}
\end{align}
with:
\begin{align}
\sigma_{ij}&=\sigma_{ij}^{(0)}\\
\alpha_{ij}&=\sigma_{ij}^{(1)}\\
\frac{\beta_{ij}}{T}&=\sigma_{ij}^{(2)}.
\end{align}
The semiclassical movement equation \eqref{semiClassicalMovement} must be used to find $\vec k(t)$. The only free paramters associated to the material under study are $E(\vec k)$ and $\tau_{\vec k}$. 
\end{tcolorbox}






\section{Simplifications}


\subsection{Equi-energy surfaces}\label{sec_surfaces}
The Dirac delta $\delta(\epsilon-E(\vec k))$ reduces the three-dimensional $\vec k$ integral to an integral over the surface where the delta's argument is zero: the surface defined by $E(\vec k)=\epsilon$ (see equation~8.57 and 8.62 of Ashcroft~\&~Mermin~\cite{ashcroft_solid_1976} for a similar statement)
\begin{align}
\iiint d^3 k 
F(\vec k)\delta\big(\epsilon - E(\vec k)\big)
&=
\iint\limits_{E(\vec k)=\text{cst.}} d^2k_{\parallel}
\int dk_{\perp}
F(\vec k)\delta\big(\epsilon - E(\vec k)\big)
\\
&=
\iint\limits_{E(\vec k)=\text{cst.}} d^2k_{\parallel} 
\int \frac{dE(\vec k)}{|\nabla_{\vec k}E(\vec k)|}
F(\vec k)\delta\big(\epsilon - E(\vec k)\big)
\\
&=
\iint\limits_{E(\vec k)=\epsilon} d^2k_{\epsilon} 
\frac{F(\vec k_{\epsilon})}{|\nabla_{\vec k}E(\vec k)|_{\vec k=\vec k_{\epsilon}}}.
% \intertext{Question: could this be written as}
% \delta\big(\epsilon-E(\vec k)\big)
% &=
% \iint\limits_{E(\vec k)=\epsilon} d^2 k_{\epsilon} \frac{\delta^{3}(\vec k-\vec k_{\epsilon})}{|\vec \nabla_{\vec k} E(\vec k)|_{\vec k=\vec k_{\epsilon}}}\ ?
\end{align}
where $\vec k_{\epsilon}$ are the vectors belonging to this equi-energy surface $E(\vec k)=\epsilon$. The integral on the Brillouin zone therefore reduces to a surface integral for each value of $\epsilon$, and \eqref{chambers1} becomes:
\begin{align}
\Aboxed{
\sigma_{ij}(\epsilon) 
&=
e^2
\iint\limits_{\substack{E(\vec k)=\epsilon\\ \text{in BZ}}}
\frac{d^2k_{\epsilon}}{(2\pi)^3}
\frac{v_{i}(\vec k_{\epsilon})\bar{v}_{j}(\vec k_{\epsilon})}{\hbar|\vec v(\vec k_{\epsilon})|}
}.
\label{chambersSurfaces}
\end{align}

\subsection{Zero field}\label{sec_zeroField}
In the limit of zero magnetic field, $\vec B\rightarrow0$, the differential equation \eqref{semiClassicalMovement} becomes trivial, and $\vec k (t)=\vec k$ at all $t$. In that case, \eqref{chambers2} simply reduces to:
\begin{align}
\bar{v}_{j}(\vec k)
&=
v_{j}(\vec k)
\int_{0}^{\infty}dt
e^{-t/\tau_{\vec k}}
\\
&=
v_{j}(\vec k)
\tau_{\vec k}.
\end{align}
However, in the case of $i\neq j$, $\sigma_{ij}(\omega)$ is zero at zero field, so we shall keep the linear term in $\vec B$ in this case (see~\cite{ziman_electrons_1960} for details). This allows to write \eqref{chambers1} as:
\begin{empheq}[box=\fbox]{align}
\lim_{\vec B\rightarrow0}\sigma_{ij}(\epsilon) 
&=
e^2\iiint_{\text{BZ}}\frac{d^3k}{(2\pi)^3}
v_{i}(\vec k)
\Bigg[
\delta_{ij}
v_{i}(\vec k)
\tau_{\vec k}
\nonumber\\
&
\phantom{=\iiint_{\text{BZ}}\frac{d^3k}{(2\pi)^3}}-
\varepsilon_{ijk}(-eB_{k})
\Big[
v_{i}(\vec k)
\frac{\partial v_{j}(\vec k)}{\partial k_j}
-v_{j}(\vec k)
\frac{\partial v_{i}(\vec k)}{\partial k_j}
\Big]
\tau_{\vec k}^2
\Bigg]
\delta\big(\epsilon-E(\vec k)\big),
\label{chambersZeroField}
\end{empheq}
where $\delta_{ij}$ is the Kronecker symbol and $\varepsilon_{ijk}$ is the Levi-Cevita symbol (which is 0 if $i=j$).

\subsection{Effective mass and Drude results}

We can carry out explicitely the integral over energy in the all-in-one equation~\eqref{chambers} and reorganize it as: 
\begin{align}
\sigma_{ij}^{(n)}
&=
-\frac{(-e)^{2-n}}{\hbar T^n}
% \int_{0}^{\infty}dt
% e^{-\int_{0}^{t}dt'/\tau_{\vec k(t')}}
\iiint_{\text{BZ}}\frac{d^3k}{(2\pi)^3}
E(\vec k)^{n}
\frac{\partial f(\epsilon)}{\partial E(\vec k)}
\frac{\partial E(\vec k)}{\partial k_i}
% \frac{\partial E(\vec k(t))}{\partial k_j}
\bar{v}_j(\vec k)
\\
&=
-\frac{(-e)^{2-n}}{\hbar T^n}
% \int_{0}^{\infty}dt
% e^{-\int_{0}^{t}dt'/\tau_{\vec k(t')}}
\iiint_{\text{BZ}}\frac{d^3k}{(2\pi)^3}
E(\vec k)^{n}
\frac{\partial f(E(\vec k))}{\partial k_i}
% \frac{\partial E(\vec k(t))}{\partial k_j}
\bar{v}_j(\vec k).
\label{checkPointMass}
\end{align}
This allow to integrate by parts as follows:
\begin{align}
\int_{-\pi/a_i}^{\pi/a_i} dk_i\ 
E(\vec k)^{n}
\frac{\partial
f(E(\vec k))
}{\partial k_i}
% \frac{\partial E(\vec k(t))}{\partial k_j}
\bar{v}_j(\vec k)
&=
\bigg[
E(\vec k)^{n}
f(E(\vec k))
% \frac{\partial E(\vec k(t))}{\partial k_j}
\bar{v}_j(\vec k)
\bigg]_{-\pi/a_i}^{\pi/a_i}
\nonumber\\
&\quad-
\int_{-\pi/a_i}^{\pi/a_i} dk_i\ 
nE(\vec k)^{n-1}
\frac{\partial E(\vec k)}{\partial k_i}
f(E(\vec k))
% \frac{\partial E(\vec k(t))}{\partial k_j}
\bar{v}_j(\vec k)
\nonumber\\
&\quad-
\int_{-\pi/a_i}^{\pi/a_i} dk_i\ 
E(\vec k)^{n}
f(E(\vec k))
% \frac{\partial^2 E(\vec k(t))}{\partial k_i\partial k_j}
\frac{\partial\bar{v}_j(\vec k)}{\partial k_i}.
\label{integrationBy3Parts}
\end{align}
Since $n$ only takes the values $0,1,2$, and under the conditions $E(-\vec k)=E(\vec k)$,
and $\bar{v}_j(-\vec k)$ = $-\bar{v}_j(\vec k)$, then the first and second terms of \eqref{integrationBy3Parts} are zero, and \eqref{checkPointMass} becomes:
\begin{align}
\Aboxed{
\sigma_{ij}^{(n)}
&=
(-e)^{2-n}
% \int_{0}^{\infty}dt
% e^{-\int_{0}^{t}dt'/\tau_{\vec k(t')}}
\iiint_{\text{BZ}}\frac{d^3k}{(2\pi)^3}
\frac{E(\vec k)^{n}}{T^n}
% \frac{\partial^2 E(\vec k(t))}{\partial k_i\partial k_j}
\frac{1}{\hbar}
\frac{\partial\bar{v}_j(\vec k)}{\partial k_i}
f(E(\vec k))
}.
\end{align}
Taking explicitely the derivative of \eqref{chambers2}:
\begin{align}
\frac{1}{\hbar}\frac{\partial\bar{v}_j(\vec k)}{\partial k_i}
&=
\int_{0}^{\infty}dt\ 
\bigg[
e^{-\int_{0}^{t}dt'/\tau_{\vec k(t')}}
\frac{1}{\hbar^2}\frac{\partial^2 E(\vec k(t))}{\partial k_i\partial k_j}
\nonumber\\
&\qquad\qquad+
e^{-\int_{0}^{t}dt'/\tau_{\vec k(t')}}
\Big(
\int_{0}^{t}\frac{dt'}{\tau_{\vec k(t')}^2}
\frac{\partial\tau_{\vec k(t')}}{\partial k_i}
\Big)
\frac{1}{\hbar^2}\frac{\partial E(\vec k(t))}{\partial k_j}
\bigg]
\end{align}
we find the effective mass:
\begin{align}
M^{-1}_{ij}(\vec k)
\equiv
\frac{1}{\hbar^2}\frac{\partial^2 E(\vec k)}{\partial k_i\partial k_j}.
\label{effectiveMass}
\end{align}
Note that if the lifetime is independent of $\vec k$, then the second term disappear, and only the first term remains. In that case, carrying the zero magnetic field manipulations of \ref{sec_zeroField} and restituting of the integral in energy lead to:
\begin{empheq}[box=\fbox]{align}
\lim_{\substack{\vec B\rightarrow 0\\ \tau_{\vec k}\rightarrow\tau}}
\sigma_{ij}^{(n)}
&=
\int_{-\infty}^{\infty}
d\epsilon
f(\epsilon)
\frac{\epsilon^{n}}{T^n}
(-e)^{2-n}
% \int_{0}^{\infty}dt
% e^{-\int_{0}^{t}dt'/\tau_{\vec k(t')}}
\iiint_{\text{BZ}}\frac{d^3k}{(2\pi)^3}
% \frac{\partial^2 E(\vec k(t))}{\partial k_i\partial k_j}
\Bigg[
\delta_{ij}M^{-1}_{ii}(\vec k)\tau
\\
&\qquad
+\varepsilon_{ijk}(-eB_k) \bigg(M^{-1}_{ii}(\vec k)M^{-1}_{jj}(\vec k) - M^{-2}_{ij}(\vec k) \bigg)\tau^2
\Bigg]
\delta(\epsilon-E(\vec k))
,
\end{empheq}
Note in particular the longitudinal electric conductivity case $\sigma_{ii} = \sigma_{ii}^{(0)}$:
\begin{align}
\lim_{\substack{\vec B\rightarrow 0\\ \tau_{\vec k}\rightarrow\tau}}
\sigma_{ii}
&=
\iiint_{\text{BZ}}\frac{d^3k}{(2\pi)^3}
f(E(\vec k))
e^2
M^{-1}_{ii}(\vec k)
\tau_{\vec k},
\end{align}
which is the Drude formula for $M_{ij}^{-1}(\vec k) = \delta_{i,j}\frac{1}{m^*_i}$:
\begin{align}
\Aboxed{
\lim_{\substack{
\vec B\rightarrow 0, 
\tau_{\vec k}\rightarrow\tau\\
M_{ij}(\vec k)\rightarrow \delta_{ij}m^*_i
}}
\sigma_{ij}
&=
\delta_{i,j}
\frac{ne^2\tau}{m^*_i}
}
\end{align}



\subsection{Zero temperature}
In the zero temperature limit, the electrical conductivity tensor \eqref{sigma} becomes: 
\begin{align}
\lim_{T\rightarrow 0}\sigma_{ij}
&=
\int_{-\infty}^{\infty} d\epsilon
\lim_{T\rightarrow 0}\Big(-\frac{\partial f(\epsilon)}{\partial\epsilon}\Big) \sigma_{ij}(\epsilon)
\\
&=
\int_{-\infty}^{\infty} d\epsilon
\delta(\epsilon-\mu) \sigma_{ij}(\epsilon)
\\
\Aboxed{
\lim_{T\rightarrow 0}\sigma_{ij}&=\sigma_{ij}(\mu).
}
\end{align}
Unfortunately, this simplification does not extend to $\tensor{\alpha}$ and $\tensor{\beta}$ for which one must instead turn to Sommerfeld expansions.
Note that together with the simplification \ref{sec_surfaces} for energy surfaces, the zero-temperature simplification implies that the conductivity at $T=0$ is entirely given by the Fermi Surface:
\begin{align}
\lim_{T\rightarrow0}\sigma_{ij} 
=\frac{e^2}{(2\pi)^3}\iint_{\text{FS}}d^2k_F
\frac{v_{i}(\vec k_F)\bar{v}_{j}(\vec k_F)}{\hbar|\vec v(\vec k_F)|}.
\label{zeroTemperature}
\end{align}

\subsection{Sommerfeld expansions}

\section{Cuprates dispersion}\label{section_disp}
\noindent
The two candidates we consider for $E(\vec k)$ are the bare dispersion $\epsilon_{\text{3D}}(\vec k)$ and the antiferromagnetisme approximation for the pseudogap $E^{\pm}_{\text{3D}}(\vec k)$ introduced in what follow.

\subsection{Bare dispersion}
To simulate overdoped cuprates, we use the one-band three-dimensional dispersion measured by ARPES~\cite{horio_three-dimensional_2018} in Eu-LSCO:
\begin{align}
\epsilon_{\text{3D}}(\vec k)=\epsilon_{\vec k}+\epsilon_{\perp}(\vec k)
\end{align}
where:
\begin{align}
\epsilon_{\vec k} 
&= 
-2t\big(\cos (k_xa) + \cos (k_yb)\big)
\nonumber\\
&\phantom{=\ }
-4t'\cos (k_xa)\cos (k_yb)
\nonumber\\
&\phantom{=\ }
-2t''\big(\cos (2k_xa)+\cos (2k_yb)\big)
\intertext{and:}
\epsilon_{\perp}(\vec k)
&=
-2t_z\cos\Big(\frac{k_xa}{2}\Big)
\cos\Big(\frac{k_yb}{2}\Big)
\cos\Big(\frac{k_zc}{2}\Big)
\nonumber\\
&\phantom{=\ }
\times\big(\cos (k_xa) - \cos (k_yb)\big)^2.
\end{align}
We use parameters $a=b=3.75$ {\AA}, $c=13.20$ {\AA}, $t=190$ meV, and $(t',t'',t_z)=(-0.14t,0.07t,0.07t)$. The chemical potential $\mu$ is obtained by solving the integral equation for a given doping $p$:
\begin{align}
p = 1 - \iiint_{BZ} \frac{d^{3}k}{4\pi^{3}} f(\epsilon_{\text{3D}}(\vec k))
\end{align}

\subsection{Pseudogap} 
To fit the results for the pseudogap regime at $p<p^*$, we start from a two-dimensional antiferromagnetic (AF) reconstruction of the bare band (following recent success of such a model for the Hall number~\cite{storey_hall_2016,verret_phenomenological_2017}):
\begin{align}
E^{\pm}(\vec k) 
= 
\tfrac{1}{2}(\epsilon_{\vec k}+\epsilon_{\vec k+\vec Q})
\pm \sqrt{
\tfrac{1}{4}(\epsilon_{\vec k}-\epsilon_{\vec k+\vec Q})^2 + M^2}
\end{align}
We cannot extend this model straightforwardly to the three-dimensional dispersion, because taking $\epsilon_{\vec k}\rightarrow\epsilon_{\text{3D}}(\vec k)$ in the above equation results in C4 rotational symmetry breaking. This is because the body-centered structure of Nd-LSCO, taken into account in $\epsilon_\perp(\vec k)$, and in which antiferromagnetism is different for $\vec Q=(\pi/a,\pi/a,0)$ and $\vec Q=(\pi/a,-\pi/a,0)$. Therefore, we will suppose that the 2D planes are independent regarding antiferromagnetism, and add the $z$-dispersion after the reconstruction:  
\begin{align}
E^{\pm}_{\text{3D}}(\vec k)=E^{\pm}(\vec k)+\epsilon_{\perp}(\vec k)
\end{align}
This may be interpreted as antiferromagnetism uncorrelated between planes.

\paragraph*{Note on numerical optimization:}
The bottleneck when evaluating equations~\eqref{zeroTemperature} at zero temperature or~\eqref{chambersSurfaces} at finite temperature is the repeated evaluation of the velocity required to solve the semiclassical equation~\eqref{semiClassicalMovement}. In practice, we can save a lot of computation time by noting that $\epsilon_{\vec k+\vec Q}$ and $\vec v(\vec k+\vec Q)$ are simply linked to $\epsilon_{\vec k}$. We first split the latter as:
\begin{align}
\epsilon_{\vec k} &= \zeta_{\vec k} + \xi_{\vec k}
\\
\xi_{\vec k}
&\equiv
-2t\big(\cos (k_xa) + \cos (k_yb)\big),
\\
\zeta_{\vec k} 
&\equiv
-4t'\cos (k_xa)\cos (k_yb)
\nonumber \\
&\phantom{=\ }
-2t''\big(\cos (2k_xa)+\cos (2k_yb)\big)
\end{align}
and note that for $\vec Q=(\pi/a,\pi/a)$:
\begin{align}
\epsilon_{\vec k+\vec Q} &= \zeta_{\vec k} - \xi_{\vec k}
\end{align}

\section{Lifetime}\label{section_lifetime}

The lifetime $\tau_{\vec k}$ functions we consider are those of previous work on Hall and Seebeck effets~\cite{verret_phenomenological_2017}, and those of the review by Hussey~\cite{hussey_phenomenology_2008}. The lifetime is the inverse of the total scattering rate:
\begin{align}
\tau_{\vec k}=\frac{\hbar}{2\Gamma(\vec k)},
\end{align}
and the scattering rate can have various contributions:
\begin{align}
\Gamma(\vec k) 
&=\eta_0
+ \eta_{\ell}|\vec v(\vec k)|
+ \eta_{N}/|\vec v(\vec k)|
\nonumber\\&
\phantom{=\ }
+ \eta_{\text{FLa}}(\omega^2 + \pi^2T^2)
+ \eta_{\text{PL}}T
\nonumber\\&
\phantom{=\ }
+ \eta_{\text{FLb}}(\omega^2 + \pi^2T^2)/|\vec v(\vec k)|
\nonumber\\&
\phantom{=\ }
+ \eta_{\text{an}}\big(\cos k_x - \cos k_y\big)^2.
\end{align}
In this expression, small numerical values might be needed to prevent divergence. Respectively taken alone, $\eta_0$ would yield constant lifetime, $\eta_\ell$ would yield constant mean-free path, $\eta_N$ would yield scattering proportional to the density of state, and $\eta_{\text{an}}$ yields a phenomenological increase of the anti-nodal scattering. There can be various reasons to consider each terms separately, or combined.

\section{strongly correlated case}

Please refer to Arsenault \emph{et al.}~\cite{arsenault_transport_2013}  for the strongly correlated case. Note that the analysis for equation 

\begin{acknowledgments}
Most of the work that lead to this note was done with Gaël Grissonanche, closely following the work of Pr. Brad Ramshaw and Yawen Fang in Cornell University.
\end{acknowledgments}


\bibliography{amro}

\end{document}

